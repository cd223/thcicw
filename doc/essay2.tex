\documentclass[11pt]{article}
%%
%% Package includes to provide the basic style
%%
\usepackage[top=3.5cm, bottom=3.5cm, left=2.5cm, right=2.5cm]{geometry}
\usepackage{hyperref}
\usepackage{pdfpages}
\usepackage{algorithm2e}
\usepackage[numbers]{natbib}
\bibliographystyle{plainnat}
\usepackage{appendix}
\usepackage{pgfplotstable}
\usepackage[font={color=darkgray,footnotesize, sf, it}]{caption}
\usepackage{amsmath}
\usepackage{helvet}
\usepackage{wrapfig}
\usepackage{multicol}
\usepackage[eulergreek]{sansmath}
\usepackage{listings}
\usepackage{setspace}
\usepackage{graphicx}
\usepackage{longtable}
\usepackage{minted}
\usepackage{relsize}
\linespread{1.2}
\lstset { %
    language=C++,
    backgroundcolor=\color{black!5}, % set backgroundcolor
    basicstyle=\ttfamily\footnotesize,% basic font setting
}
\usepackage{bchart}
\usepackage{booktabs}
\usepackage{xcolor}

\definecolor{skyblue}{RGB}{203,240,255}
\definecolor{forestgreen}{RGB}{0,96,50}
\definecolor{indigo}{RGB}{4,87,239}
\definecolor{darkindigo}{RGB}{1,40,112}
\definecolor{darkpurple}{RGB}{32,14,104}
\definecolor{apple}{RGB}{0,158,13}
\definecolor{bad}{RGB}{239,55,55}
\definecolor{badtoavg}{RGB}{247,128,98}
\definecolor{avg}{RGB}{255,218,117}
\definecolor{avgtogood}{RGB}{190,247,140}
\definecolor{good}{RGB}{102,226,102}

\setlength{\parindent}{0em}
\setlength{\parskip}{1em}
\setlength\parindent{0pt}

\usepackage{titlesec}

\titlespacing\section{0pt}{12pt plus 4pt minus 2pt}{0pt plus 2pt minus 2pt}
\titlespacing\subsection{0pt}{12pt plus 4pt minus 2pt}{0pt plus 2pt minus 2pt}
\titlespacing\subsubsection{0pt}{12pt plus 4pt minus 2pt}{0pt plus 2pt minus 2pt}

\date{22-03-2019}
\author{cjd47}
%%
%% END OF DEFINITIONS
%%

\begin{document}
\essaytitle{Critical Review: Exploring Interactions with Physically Dynamic Bar Charts}

\sectiontitle{Introduction}
Studies investigating how data can be effectively presented to, explored and interpreted by users forms the core part of Information Visualisation (`InfoVis') to support users in the decision-making process. This review summarises and critical analyses \citet{taher2015} whose paper explores the use of physically dynamic bar chart as a device for exploring user interactions with visualisations of data, to determine future work in this domain of Information Visualisation.

\sectiontitle{Summary of Contributions}
\citeauthor{taher2015} seeks to extend existing work on use of physical visualisations (\textit{physicalizations}) to investigate how users interact with \textit{physically dynamic} bar charts as a way of exploring and manipulating shape-changing datasets in the physical world. Much of the existing work reliant on use of physicalizations involve \textit{static} models that do not respond to user interactions. With the advent of shape-changing technology, there is a window of opportunity for physically dynamic displays to help decision makers reason about and manipulate data sets.

The point system described by the article is EMERGE - a $10\times10$ set of dynamic actuator rods with an RGB display projected onto it (Figure \ref{fig:taher2015-emerge}). This system allows users to interact with the dataset it represents using four task-sets derived from the taxonomy of interactive dynamic described by \citet{heer2012}.

\begin{figure}[H]
\centering
\includegraphics[width=0.5\textwidth]{img/taher2015-emerge.png} 
\caption{EMERGE: Exploring Interactions with Physically Dynamic Bar Charts using actuating physical rods and RGB LEDs to display international export data.}\label{fig:taher2015-emerge}
\end{figure}

\begin{figure}[H]
\centering
\includegraphics[width=0.8\textwidth]{img/heer2012-taxonomy.png} 
\caption{Taxonomy of interactive dynamics described by \protect\citet{heer2012}.}\label{fig:heer2012-taxonomy}
\end{figure}

\begin{table}[H]
\centering
\caption{Task-sets and interaction techniques explored during the user study.}\label{tbl:taher2015-user-study}
\includegraphics[width=0.65\textwidth]{img/taher2015-user-study.png} 
\end{table}

\sectiontitle{Justifications for Conclusions}
...

\begin{figure}[H]
\minipage{0.5\textwidth}
\centering
  \includegraphics[height=3.5cm]{img/taher2015-annotation.png}
  \caption{Annotation (Point technique).}\label{fig:taher2015-annotation}
\endminipage\hfill
\minipage{0.5\textwidth}%
\centering
  \includegraphics[height=3.5cm]{img/taher2015-organize.png}
  \caption{Organisation (Drag and Drop technique).}\label{fig:taher2015-organize}
\endminipage
\end{figure}

\begin{figure}[H]
\centering
\includegraphics[width=0.5\textwidth]{img/taher2015-likert.png} 
\caption{Likert scale ratings for helpfulness of interaction
techniques. Range = 1: Strongly Disagree, 5: Strongly Agree.}\label{fig:taher2015-likert}
\end{figure}

\sectiontitle{Limitations and Suggested Further Work}
...

\sectiontitle{Conclusion}
...

\wordcount{0}

\newpage
\small
\bibliography{bib2}
\normalsize
\end{document}